\clearpage

\section{Gaussian Source}

\begin{tcolorbox}	
	\begin{tabular}{p{2.75cm} p{0.2cm} p{10.5cm}} 	
		\textbf{Header File}   &:& gaussian\_source.h \\
		\textbf{Source File}   &:& gaussian\_source.cpp \\
	\end{tabular}
\end{tcolorbox}

This block simulates a random number generator that follows a Gaussian statistics. It produces one output real signal and it doesn't accept input signals.

\subsection*{Input Parameters}

\begin{table}[h]
	\centering
	\begin{tabular}{|c|c|c|c|cccc}
		\cline{1-4}
		\textbf{Parameter} & \textbf{Type} & \textbf{Values} &   \textbf{Default}& \\ \cline{1-4}
		mean & double & any & $0$ \\ \cline{1-4}
		Variance & double & any & $1$ \\ \cline{1-4}
	\end{tabular}
	\caption{Gaussian source input parameters}
	\label{table_Gaussian_Source}
\end{table}


\subsection*{Methods}

GaussianSource() {}
\bigbreak
GaussianSource(vector$<$Signal *$>$ \&InputSig, vector$<$Signal *$>$ \&OutputSig) :Block(InputSig, OutputSig)\{\};
\bigbreak
void initialize(void);
\bigbreak
bool runBlock(void);
\bigbreak
void setAverage(double Average) ;

\subsection*{Functional description}

This block generates a complex signal with a specified phase given by the input parameter \textit{phase}.

\pagebreak
\subsection*{Input Signals}

\subparagraph*{Number:} 0

\subsection*{Output Signals}

\subparagraph*{Number:} 1

\subparagraph*{Type:} Continuous signal (TimeDiscreteAmplitudeContinuousReal)

\subsection*{Examples}

\subsection*{Sugestions for future improvement}


