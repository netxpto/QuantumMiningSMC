\clearpage

\section{Cloner}
\label{sec:cloner}
\begin{refsection}

\begin{tcolorbox}	
\begin{tabular}{p{2.75cm} p{0.2cm} p{10.5cm}} 	
\textbf{Header File}    &:& cloner\_*.h \\
\textbf{Source File}    &:& cloner\_*.cpp \\
\textbf{Version}        &:& 20190114 (Daniel Pereira)
\end{tabular}
\end{tcolorbox}

\subsection*{Input Parameters}

This block takes no input parameters.


\subsection*{Methods}

\begin{itemize}
  \item Cloner(vector<Signal *> \&InputSig, vector<Signal *> \&OutputSig) :Block(InputSig,OutputSig)\{\};
  \item void initialize(void);
  \item bool runBlock(void);
\end{itemize}




\subsection*{Input Signals}

\textbf{Number}: 1\\
\textbf{Type}: Real, Complex, Complex\_XY, Binary


\subsection*{Output Signals}

\textbf{Number}: Arbitrary\\
\textbf{Type}: Same as input

\subsection*{Functional Description}

This block accepts a signal and outputs a number of copies of the input. The number of the copies is set by the number of output signals given to the block. The block adapts dynamically.

\subsection*{Theoretical Description}\label{bercalc}


% bibliographic references for the section ----------------------------
\clearpage
\printbibliography[heading=subbibliography]
\end{refsection}
\addcontentsline{toc}{subsection}{Bibliography}
\cleardoublepage
% --------------------------------------------------------------------- 