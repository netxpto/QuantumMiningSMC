\clearpage

\section{Clock}

\begin{tcolorbox}	
	\begin{tabular}{p{2.75cm} p{0.2cm} p{10.5cm}} 	
		\textbf{Header File}   &:& clock.h \\
		\textbf{Source File}   &:& clock.cpp \\
	\end{tabular}
\end{tcolorbox}

This block doesn't accept any input signal. It outputs one signal that corresponds to a sequence of Dirac's delta functions with a user defined \textit{period}.

\subsection*{Input Parameters}

%\begin{itemize}
%	\item period\{ 0.0 \};
%	\item samplingPeriod\{ 0.0 \};
%\end{itemize}

\begin{table}[h]
	\centering
	\begin{tabular}{|c|c|c|c|cccc}
		\cline{1-4}
		\textbf{Parameter} & \textbf{Type} & \textbf{Values} &   \textbf{Default}& \\ \cline{1-4}
		period & double & any & $0.0$ \\ \cline{1-4}
		samplingPeriod & double & any & $0.0$ \\ \cline{1-4}
	\end{tabular}
	\caption{Binary source input parameters}
	\label{table:clock_in_par}
\end{table}

\subsection*{Methods}

Clock() {}
\bigbreak
Clock(vector$<$Signal *$>$ \&InputSig, vector$<$Signal *$>$ \&OutputSig) :Block(InputSig, OutputSig) {}
\bigbreak
void initialize(void)
\bigbreak
bool runBlock(void)
\bigbreak
void setClockPeriod(double per)
\bigbreak
void setSamplingPeriod(double sPeriod)

\subsection*{Functional description}


\pagebreak

\subsection*{Input Signals}

\subparagraph*{Number:} 0

\subsection*{Output Signals}

\subparagraph*{Number:} 1

\subparagraph*{Type:} Sequence of Dirac's delta functions. (TimeContinuousAmplitudeContinuousReal)

\subsection*{Examples}

%\begin{figure}[h]
%	\centering
%	\includegraphics[width=\textwidth]{./lib/clock/figures/Clock_output}
%	\caption{Example of the output signal of the clock}\label{Clock_output}
%\end{figure}

\subsection*{Sugestions for future improvement}

