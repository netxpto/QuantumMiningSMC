\clearpage

\section{Bit Error Rate}
\label{sec:bit_error_rate}
\begin{refsection}

\begin{tcolorbox}	
\begin{tabular}{p{2.75cm} p{0.2cm} p{10.5cm}} 	
\textbf{Header File}    &:& bit\_error\_rate\_*.h \\
\textbf{Source File}    &:& bit\_error\_rate\_*.cpp \\
\textbf{Version}        &:& 20171810 (Daniel Pereira)\\
                        &:& 20181424 (Mariana Ramos)
\end{tabular}
\end{tcolorbox}

\subsection*{Input Parameters}

\begin{table}[H]
\centering
\begin{tabular}{|l|l|l|}
\hline
Name           & Type           & Default Value     \\ \hline
alpha          & double         & 0.05              \\ \hline
m              & integer        & 0                 \\ \hline
lMinorant      & double         & $1\times10^{-10}$ \\ \hline
\end{tabular}
\end{table}


\subsection*{Methods}

\begin{itemize}
  \item BitErrorRate(vector<Signal *> \&InputSig, vector<Signal *> \&OutputSig) :Block(InputSig,OutputSig)\{\};
  \item void initialize(void);
  \item bool runBlock(void);
  \item void setConfidence(double P) \{ alpha = 1-P; \}
  \item void setMidReportSize(int M) \{ m = M; \}
  \item void setLowestMinorant(double lMinorant) \{ lowestMinorant=lMinorant; \}
\end{itemize}




\subsection*{Input Signals}

\textbf{Number}: 2\\
\textbf{Type}: Binary (DiscreteTimeDiscreteAmplitude)


\subsection*{Output Signals}

\textbf{Number}: 1\\
\textbf{Type}: Binary (DiscreteTimeDiscreteAmplitude)

\subsection*{Functional Description}

This block accepts two binary strings and outputs a binary string, outputting a 1 if the two input samples are equal to each other and 0 if not. This block also outputs \textit{.txt} files with a report of the estimated Bit Error Rate (BER), $\widehat{\text{BER}}$ as well as the estimated confidence bounds for a given probability $\alpha$. In version \textbf{20181113} instead of the previous binary output string, this block outputs a 0 if the two input samples are equal and 1 if not.
\par
The block allows for mid-reports to be generated, the number of bits between reports is customizable, if it is set to 0 then the block will only output the final report. In version \textbf{20180424} this block can operate mid-reports using a CUMULATIVE mode, in which the BER is calculated in a cumulative way taking into account all received bits, coincidences and errors, or in a RESET mode, in which at each \textbf{m} bits the number of received bits and coincidence bits is reset for the BER calculation.

\subsection*{Theoretical Description}\label{bercalc}
The $\widehat{\text{BER}}$ is obtained by counting both the total number received bits, $N_T$, and the number of coincidences, $K$, and calculating their relative ratio:
\begin{equation}
\widehat{\text{BER}}=1-\frac{K}{N_T}.
\end{equation}
The upper and lower bounds, $\text{BER}_\text{UB}$ and $\text{BER}_\text{LB}$ respectively, are calculated using the Clopper-Pearson confidence interval, which returns the following simplified expression for $N_T>40$~\cite{Almeida16}:
\begin{align}
\text{BER}_\text{UB}&=\widehat{\text{BER}}+\frac{1}{\sqrt{N_T}}z_{\alpha/2}\sqrt{\widehat{\text{BER}}(1-\widehat{\text{BER}})}+\frac{1}{3N_T}\left[2\left(\frac{1}{2}-\widehat{\text{BER}}\right)z_{\alpha/2}^2+(2-\widehat{\text{BER}})\right]\\
\text{BER}_\text{LB}&=\widehat{\text{BER}}-\frac{1}{\sqrt{N_T}}z_{\alpha/2}\sqrt{\widehat{\text{BER}}(1-\widehat{\text{BER}})}+\frac{1}{3N_T}\left[2\left(\frac{1}{2}-\widehat{\text{BER}}\right)z_{\alpha/2}^2-(1+\widehat{\text{BER}})\right],
\end{align}
where $z_{\alpha/2}$ is the $100\left(1-\frac{\alpha}{2}\right)$th percentile of a standard normal distribution.

\subsection*{Version 20181424}

Version 20181424 allows the user to choose the type of middle reports he wants. So, the input parameter \textrm{mideRepType} can has the value \textit{Cumulative}, where the BER estimation is done by taking into account all samples acquired in a cumulative way, or \textit{Reset}, where the BER estimation is done by taking into account only the number of samples set as \textit{m} in each middle report.

\begin{itemize}
  \item \textbf{Input Parameters}
  \begin{table}[H]
    \centering
    \begin{tabular}{|l|l|l|}
    \hline
    Name           & Type           & Default Value     \\ \hline
    midRepType     & MidReportType  & Cumulative \\ \hline
    \end{tabular}
  \end{table}

  \item \textbf{Methods}
  \begin{itemize}
    \item void setMidReportType(MidReportType mrt) \{ midRepType = mrt; \};
  \end{itemize}
\end{itemize}




% bibliographic references for the section ----------------------------
\clearpage
\printbibliography[heading=subbibliography]
\end{refsection}
\addcontentsline{toc}{subsection}{Bibliography}
\cleardoublepage
% --------------------------------------------------------------------- 